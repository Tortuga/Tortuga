%		TODO:
%		- simplifier les r�gles


\documentclass[a4paper, 11pt]{report}

\usepackage[T1]{fontenc}
\usepackage[latin9]{inputenc}
\usepackage[frenchb]{babel} %ou \usepackage[francais]{babel} 
\usepackage{url} %�crire des adresses cliquables
\usepackage{lmodern} %changer pack de police
\usepackage[top=3cm, bottom=3cm, left=2.7cm, right=2.7cm]{geometry} %g�rer les marges
\usepackage{color}
\usepackage{mathtools}
\usepackage[babel=true]{csquotes} % csquotes va utiliser la langue d�finie dans babel
\usepackage{graphicx}
\DeclareGraphicsExtensions{.jpeg, .png , .gif, .bmp}

%%%%%%%%%%%%%%%%%%%%%%%%%
\begin{document}

\author{DUT Informatique - Projet Tortuga}
\title{\textbf{Manuel Utilisateur}}
\date{2011-2012}
\maketitle
%%%%%%%%


\chapter*{Introduction}

Bienvenue dans le manuel utilisateur de l'application Tortuga ! \\

\tableofcontents			
 
 
\chapter{Les r�gles du jeu}

\section{R�gles de bases}

\begin{itemize}
\item Les tortues se d�placent (avance ou saute) \textbf{toujours} vers l'avant (tout droit ou de cot�), mais jamais vers l'arri�re
\item Lorsqu'une tortue en saute une autre :
	\begin{itemize}
	\item Si c'est une de ses tortues, rien de sp�cial ne se passe.
	\item Si c'est une tortue adverse, la tortue adverse est retourn�e est devient neutre (d�sactiv�e).
	\item Si c'est une tortue neutre, celle-ci redevient active et le joueur qui a r�alis� le saut d�cide de sa nouvelle �quipe.
	\end{itemize}
\item Un joueur gagne en d�sactivant toutes les tortues adverses ou en placant une de ses tortues sur la case "Home" de son adversaire :
\end{itemize}
\vspace{0.3cm}
\begin{center}
	\includegraphics[scale=0.5]{screen.png}
\end{center}

\begin{itemize}
\item Lorsque l'on peut sauter une tortue adverse, ce saut est obligatoire.
\item Lorsque l'on peut sauter une tortue neutre, ce saut n'est pas obligatoire !
\item On ne peut pas arr�ter un saut tant qu'un autre est possible, les sauts entam�s doivent absolument arriver � leur terme (m�me si les tortues � sauter sont neutres\dots{})
\item On ne peut pas sauter une de ses tortues puis une tortue adverse (et inversement).
\item Impossible de revenir en arri�re, m�me lors d'un saut !
\item Apr�s le saut d'une tortue neutre, si l'on a le choix, on peut choisir entre sauter une de ses tortues ou une tortue adverse.
\end{itemize}

\section{Variante mode �closion}


\chapter{Comment jouer ?}

\section{Partie contre un autre joueur}

\subsection{Partie simple}
\subsection{Partie avec variante}

\section{Partie contre une IA}

\subsection{Partie simple}
\subsection{Partie avec variante}


\chapter{R�cup�rer l'application et sp�cifications minimum}

\section{Application sur ordinateur}
\section{Application mobile}


\chapter{F.A.Q.}


			
\end{document}
